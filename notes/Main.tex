% main.tex - Master's Thesis Template
% University of Oslo - Department of Psychology
% Cognitive Neuroscience Master's Programme

\documentclass[12pt,a4paper]{report}

% ============= PACKAGES =============
\usepackage[utf8]{inputenc}
\usepackage[english]{babel}
\usepackage{times} % Times New Roman font
\usepackage[margin=2.5cm]{geometry} % 2.5cm margins as required
\usepackage{setspace}
\onehalfspacing % 1.5 line spacing as required

% Graphics and figures
\usepackage{graphicx}
\usepackage{float}

\usepackage{subfigure}

% Tables
\usepackage{booktabs}
\usepackage{tabularx}
\usepackage{multirow}


% Math and algorithms
\usepackage{amsmath}
\usepackage{amssymb}
\usepackage{algorithm}
\usepackage{algorithmic}

% Citations and references (APA 7)
\usepackage[style=apa,backend=biber]{biblatex}
\addbibresource{references.bib}

% Links and cross-references
\usepackage{hyperref}
\hypersetup{
    colorlinks=true,
    linkcolor=black,
    filecolor=black,      
    urlcolor=blue,
    citecolor=black
}

% Code listings (for Python/MATLAB code)
\usepackage{listings}
\lstset{
    basicstyle=\footnotesize\ttfamily,
    breaklines=true,
    frame=single,
    numbers=left,
    numberstyle=\tiny,
    captionpos=b
}

% Appendices
\usepackage[toc,page]{appendix}

% Custom commands for consistency
\newcommand{\todo}[1]{\textcolor{red}{[TODO: #1]}}
\newcommand{\note}[1]{\textcolor{blue}{[NOTE: #1]}}

% Simple chapter formatting - normal size headings with reasonable spacing
\usepackage{titlesec}
\titleformat{\chapter}
  {\normalfont\LARGE\bfseries}{}{ 0pt}{\LARGE}
\titlespacing*{\chapter}{0pt}{20pt}{20pt}


% ============= DOCUMENT =============
\begin{document}

% ============= FRONT MATTER =============
\begin{titlepage}
    \centering
    \includegraphics[width=0.3\textwidth]{Master thesis/figures/uio_logo.png}
    \vspace{2cm}
    
    {\LARGE\bfseries Sleeping patterns and flexible minds:
    
    Disentangling the effects of habitual and nightly sleep on cognitive control\par}
    
    \vspace{2cm}
    
    {\large Røskva Bjørgfinsdóttir\par}
    
    \vspace{3cm}
    
    {\large Master thesis in cognitive neuroscience\\
    Department of Psychology\\
    University of Oslo\par}
    
    \vspace{2cm}
    
    {\large Spring 2026\par} % Update semester/year
    
    \vfill
    
    {\large Supervisors:\\ 
    Christina Thunberg\\
    René Huster\par}
\end{titlepage}

% Abstract (max 500 words, on one page)
\newpage
\thispagestyle{empty}
\begin{center}
    \Large\textbf{Abstract}
\end{center}

\noindent
\textbf{Author:} Røskva Bjørgfinsdóttir\\
\textbf{Title:} Sleeping patterns and flexible minds: Disentangling the effects of habitual and nightly sleep on cognitive control\\
\textbf{Supervisor:} Christina Thunberg

\vspace{0.5cm}

For my master thesis I'm using data from the Mind the Sleep project initiated by Margrethe Hansen and René Huster, as a part of Margrethe’s PhD project. The main aim of the study was to explore how sleep disturbances impact executive functions and affect (with a particular focus on depressive symptoms), as well as how it might relate to brain function.

\paragraph{}
Here are some title suggestions:

\paragraph{Sleeping patterns and flexible minds:} Disentangling the effects of habitual and nightly sleep on cognitive control
dynamic aspects of sleep and cognitive flexibility, shifting


\textbf{Keywords:} sleep, executive function, depression, actigraphy, cognitive performance

% Table of Contents
\newpage
\tableofcontents

% List of Figures and Tables (optional but recommended)
%\newpage
%\listoffigures
%\listoftables

% ============= MAIN CONTENT =============
\chapter{Introduction}

Sleep is a fundamental biological process that supports cognitive and emotional functioning. Sleep is not a unitary construct, nor is it stable from night to night. Even in healthy adults, sleep duration and quality vary across individuals and across time, shaped by circadian rhythms, lifestyle factors, and environmental demands \parencite{foster2020sleep}. This variability is increasingly recognized as informative rather than merely ``noise''. Habitual sleep patterns may reflect relatively stable individual differences, whereas nightly deviations can reflect transient changes in arousal, fatigue, and regulatory capacity \parencite{fjell2024sleep}. Understanding how these different aspects of sleep relate to cognition is important because everyday performance often depends on long-term sleep habits as well as how well one slept in the days leading up to a demanding situation.

A large body of research indicates that sleep disturbances are associated with impairments in executive functions, particularly processes often described as cognitive control \parencite{sen2023sleepef,tai2022sleepef}. However, the relationship between sleep and executive functioning appears to be selective rather than global. Experimental sleep deprivation does not uniformly impair all cognitive operations. Studies show that it tends to disrupt top-down control processes more strongly than relatively automatic or stimulus-driven processes \parencite{kusztor2019differentialsleep}. This pattern suggests that sleep may be especially relevant for cognitive control demands that require maintaining and updating task goals, resolving interference, and flexibly adjusting behavior to changing task requirements.

One central component of cognitive control is cognitive flexibility, the ability to adapt behaviour when goals or task demands change. Contemporary frameworks emphasize that flexibility is not simply the opposite of stability, but rather reflects coordinated processes that regulate when to maintain a goal and when to update it \parencite{egner2023principles}. In everyday life, cognitive flexibility supports switching between rules, adapting to new information, and shifting attention between competing task sets. When flexibility fails, behaviour becomes rigid, error-prone, or inefficient, particularly in situations that require rapid reconfiguration of task goals and response mappings.

Cognitive flexibility is commonly operationalized using task-switching paradigms. In such tasks, participants alternate between two simple task rules and must respond according to the currently relevant rule. Performance is typically slower and less accurate on switch trials compared with repeat trials, reflecting the cost of reconfiguring task sets and resolving interference from the previously active rule \parencite{monsell2003switching,kiesel2010review}. Task-switching tasks can also be designed to include congruent and incongruent trials, where the same stimulus affords either compatible or competing response mappings across task rules. These congruency manipulations provide an additional window into interference control within task switching and can help distinguish between general switching costs and stimulus-based conflict effects \parencite{kiesel2010review}.

In addition to behavioural measures such as reaction time and accuracy, cognitive control can be indexed using electrophysiological markers. A robust and widely studied neural correlate of cognitive control is frontal midline theta (FM-theta), a rhythmic activity in the theta band that is typically observed over midline frontal electrodes and has been linked to conflict monitoring, error processing, and the engagement of control \parencite{cavanagh2014theta}. FM-theta tends to increase when control demands are higher, including during interference and task switching, and it is often interpreted as reflecting control-related signalling within medial frontal systems \parencite{depestele2023thetadriving}. Importantly for the present thesis, sleep loss has been associated with reduced control-related neural responses, including attenuated theta activity during demanding tasks \parencite{kusztor2019differentialsleep,wilckens2014sleeppreparation}. This motivates examining whether naturally occurring variation in sleep relates to behavioural indices of flexibility and to neural indices of cognitive control engagement.

Despite the substantial literature linking sleep and executive function, two conceptual gaps are particularly relevant. First, many studies treat sleep as a single predictor (e.g., ``sleep duration'') rather than a multidimensional phenomenon encompassing duration, efficiency, continuity, and variability. This can obscure meaningful differences between individuals who sleep the same number of hours but differ in sleep fragmentation or efficiency. Second, even when objective sleep measures are available, fewer studies explicitly distinguish between habitual sleep level (how a person typically sleeps over days or weeks) and sleep variability (how much a person's sleep fluctuates from night to night). These two aspects may have different cognitive correlates. Habitual sleep may capture more stable sleep-related characteristics, whereas variability may reflect dysregulation or instability in sleep timing and quality that could be relevant for cognitive control \parencite{fjell2024sleep}.

The present thesis addresses these gaps by examining how objectively measured sleep patterns relate to cognitive flexibility and its neural correlates. Using data from the Mind the Sleep project, sleep is quantified over four weeks using actigraphy, allowing estimation of both habitual sleep level (e.g., mean sleep duration and efficiency) and sleep variability (e.g., night-to-night fluctuations in these measures). Cognitive flexibility is assessed with a task-switching paradigm performed during an EEG session, enabling analysis of both behavioural performance and FM-theta activity during task switching. By combining objective sleep assessment with behavioural and electrophysiological indices of cognitive control, the thesis aims to provide a nuanced account of how sleep patterns relate to flexible cognition in adults.

\section{Research Question and Hypotheses}

The overarching research question is:

\begin{quote}
To what extent do habitual sleep level and sleep variability predict cognitive flexibility and frontal midline theta during a task-switching paradigm?
\end{quote}

Based on prior work linking sleep disturbances to impaired top-down control and altered control-related neural responses \parencite{kusztor2019differentialsleep,wilckens2014sleeppreparation}, the following hypotheses are proposed. First, poorer habitual sleep (e.g., shorter average sleep duration and lower average sleep efficiency) is expected to be associated with reduced cognitive flexibility, reflected in larger switch costs and/or stronger interference effects. Second, greater sleep variability is expected to be associated with poorer cognitive flexibility, above and beyond habitual sleep level, reflecting less stable cognitive control capacity. Third, both poorer habitual sleep and greater sleep variability are expected to be associated with altered FM-theta during task switching, consistent with sleep-related differences in control engagement \parencite{cavanagh2014theta,depestele2023thetadriving}. Finally, if FM-theta reflects control recruitment during demanding trials, individual differences in behavioural costs (e.g., switch costs) are expected to covary with theta indices of control engagement.

Together, these analyses aim to clarify how multidimensional sleep patterns relate to flexible cognition and its neural signatures, providing insight into whether both the level and the stability of sleep are relevant for cognitive control in everyday conditions.


\chapter{Background}

A student walks into a busy café, orders a cappuccino, and sits down to read a demanding article about cognitive control. Around him, conversations overlap, chairs scrape against the floor, and children move unpredictably between tables. Although he initially enjoys the lively atmosphere, he soon needs to suppress the surrounding distractions in order to focus on the text. When the waitress arrives with his coffee, he briefly shifts his attention to thank her before returning to reading. In this seemingly ordinary situation, adapted from \textcite{egner2023principles}, the student relies on a set of mental operations that allow him to maintain task goals, prioritize relevant information, inhibit competing inputs, and adjust behaviour as circumstances change. These abilities are commonly described as \emph{cognitive control} or \emph{executive functions}.

Such control processes are important both for our daily functioning and for more demanding and novel situations. When they fail, the consequences can extend far beyond a minor faux pas, like neglecting to thank the waitress for her service. Lapses in attention, difficulties updating behaviour, or an inability to suppress irrelevant information can impair decision-making in contexts ranging from academic performance to clinical judgement and traffic safety. Understanding the mechanisms that support flexible, goal-directed behaviour is therefore a central aim in cognitive neuroscience.

\section{Frameworks of cognitive control}

Executive functions are often conceptualized as higher-order processes that enable individuals to regulate thoughts and actions in accordance with internal goals. A highly influential account is the unity and diversity framework proposed by \textcite{friedman2017unity}, which reconciles two seemingly competing views: that executive functions reflect a common control capacity, and that they consist of partially separable components. According to this model, tasks requiring inhibition, shifting, and updating share variance because they draw on a domain-general ability to actively maintain goals and bias processing toward task-relevant information. At the same time, each component captures additional processes that are not fully reducible to this shared control factor. Shifting tasks, for example, involve mechanisms related specifically to disengaging from one task set and activating another.

This perspective implies that cognitive control is neither a single unified faculty nor a collection of entirely independent skills. Rather, it reflects coordinated processes that must operate together to support adaptive behaviour. This coordination becomes visible in situations that require balancing persistence with adaptability, such as maintaining a current goal (reading an article) while remaining prepared to modify it when the environment changes (saying thank you for coffee).

Building on this idea, contemporary accounts emphasize that effective cognition depends on a dynamic interplay between stability and flexibility. Rather than representing opposite ends of a continuum, these properties can be understood as complementary demands that must be regulated in accordance with situational requirements \parencite{egner2023principles}. Returning to the café example, the student must sustain attention on the article despite ongoing distraction, demonstrating stability. Yet he must also remain capable of shifting attention when the waitress approaches, reflecting flexibility. Adaptive behaviour emerges from efficiently coordinating when to maintain a goal state and when to update it.
Mechanistically, this coordination has been described in terms of biasing and gating processes within working memory. Biasing prioritizes currently relevant representations, protecting them from interference, whereas gating determines when new information is allowed to enter and update the active task set. Cognitive flexibility therefore depends on the ability to regulate control processes in a context-sensitive manner, rather than just responding quickly to every stimulus.

Task-switching paradigms are a well-established way to operationalize this form of flexibility. In such paradigms, participants alternate between two task rules and must select responses according to the currently relevant rule. Performance is typically slower and less accurate on switch trials than on repeat trials. This phenomenon is known as the switch cost, and is thought to reflect the cognitive demands associated with reconfiguring task sets and overcoming interference from the previously active rule \parencite{monsell2003switching,kiesel2010review}. Many task-switching designs further manipulate stimulus congruency, such that some stimuli afford the same response across task rules whereas others activate competing responses. These congruency effects provide an additional window into interference control within shifting contexts, highlighting that flexibility involves both updating task representations and resolving conflict at the level of stimulus–response mappings.

Beyond behavioural performance, cognitive control can also be examined through its neural signatures. One of the most robust electrophysiological markers associated with control demands is frontal midline theta (FM-theta), a rhythmic activity observed over medial frontal scalp regions and commonly linked to generators in the anterior cingulate and medial prefrontal cortex \parencite{cavanagh2014theta}. FM-theta reliably increases during conditions that require enhanced control, including conflict processing, error monitoring, and task switching \parencite{depestele2023thetadriving}. It is often interpreted as reflecting a control-related signal that supports the coordination of task-relevant processing when demands are high.

Taken together, these frameworks characterize cognitive flexibility as the capacity to regulate the balance between maintaining and updating goal representations in the face of changing demands. Behavioural measures derived from task-switching paradigms capture the efficiency of this regulation, while FM-theta provides a complementary neural index of control engagement. This theoretical foundation is central for understanding how factors that influence cognitive state, like sleep, could shape flexible cognition.


\begin{figure}[h!]
    \centering
    \includegraphics[width=0.5\textwidth]{Master thesis/figures/theory/Egner2023fig1WM.jpg}
    \caption{Egner's (2023) figure of working memory.}
    \label{fig:mylabel}
\end{figure}

The dual mechanisms of control (DMC, \cite{braver2012dualmechanisms}) framework offers a mechanistic account of executive function by distinguishing between two temporal modes of control, proactive and reactive. Proactive control refers to the sustained, anticipatory maintenance of task goals to bias perception and action before an event occurs, whereas reactive control reflects a just-in-time reactivation of goals when interference or conflict is detected. These modes differ in timing and in cognitive demands. Proactive control is more effortful and resource-dependent, and reactive control is relatively low-cost and stimulus-driven. If you look at the unity–diversity model in relation to DMC, it highlights that the same executive task can rely on multiple underlying control modes, and that individuals may differ in the extent to which they engage proactive versus reactive strategies. 

This distinction could help us understand variability in executive functioning across time. Because proactive control depends on sustained goal maintenance, it is more vulnerable to fluctuations in arousal, attentional capacity, and fatigue, whereas reactive control tends to remain stable under such conditions. In this context, sleep becomes a theoretically meaningful source of variation, like we saw in the study of \cite{kusztor2019differentialsleep}. Poor or irregular sleep impaired proactive, top-down control processes more strongly than automatic, reactive ones. This could potentially reveal something about which components of a task reflect stable trait-like abilities and which reflect state-dependent fluctuations. Thus, the DMC framework can be used to clarify some of the mechanisms underlying executive performance and provide a principled basis for predicting how habitual and nightly sleep variability should differentially influence the processes engaged during tasks such as task switching.

Considered alongside the unity–diversity framework, the DMC account suggests that executive tasks rarely rely on a single control process. Instead, performance reflects the extent to which individuals engage proactive versus reactive strategies, as well as their ability to flexibly shift between them. Two individuals may therefore achieve similar behavioural outcomes through partially different control dynamics. Sleep variability may contribute to systematic fluctuations in how cognitive control is implemented rather than simply degrading performance overall.


\subsection{Task switching and set shifting}

Task-switching paradigms are widely used to investigate cognitive flexibility because they capture the ability to disengage from one task set and configure another in response to changing demands \parencite{monsell2003switching,kiesel2010review}. In these paradigms, participants alternate between simple classification rules, typically producing slower and less accurate responses on switch trials than on repetition trials. This switch cost is generally interpreted as reflecting the cognitive effort required to reconfigure task representations and overcome interference from the previously relevant rule.

Many task-switching designs also manipulate stimulus congruency. Congruent stimuli afford the same response across task rules, whereas incongruent stimuli activate competing response mappings. Congruency effects therefore provide an additional measure of interference control within shifting contexts. Interference is often amplified on switch trials, suggesting that the need to update task sets may temporarily increase susceptibility to competing stimulus–response associations \parencite{kiesel2010review}.

From the perspective of the dual mechanisms framework, successful task switching is thought to rely on the ability to maintain task goals while flexibly resolving interference when competing responses are activated. In paradigms where the upcoming rule is not explicitly cued prior to stimulus onset, opportunities for advance preparation are limited, increasing the importance of reactive control processes. Participants must therefore retrieve and implement the relevant task set at the moment the stimulus appears, placing greater demands on conflict monitoring and rapid adjustment of behaviour. Task-switching paradigms of this kind are well suited for investigating factors that may influence how effectively cognitive control is maintained and recruited when advance preparation is limited. Under such conditions, successful performance may depend on the stability of top-down control processes, which have been shown to be sensitive to sleep-related disruptions. TVINGER JEG FREM NOE SOM IKKE STEMMER HER?


\paragraph{Neural correlates of cognitive flexibility}

At the neural level, cognitive flexibility is supported by a distributed frontoparietal control network involving medial and lateral prefrontal regions as well as posterior parietal cortex \parencite{kiesel2010review}. These regions are broadly associated with goal maintenance, attentional allocation, and the resolution of interference, which is needed when behaviour must be reconfigured quickly in response to changing task demands. Successful set shifting depends on the ability to represent multiple task rules, as well as mechanisms that monitor conflict and signal the need for increased control.

Electrophysiological research provides converging evidence for such mechanisms. Frontal midline theta (FM-theta) is a rhythmic activity typically observed in the 4–8 Hz range over medial frontal scalp sites. It has been suggested as a robust neural marker of cognitive control. Convergent findings from EEG and intracranial recordings suggest that FM-theta is primarily generated in the anterior cingulate cortex (ACC) and adjacent medial prefrontal cortex (mPFC), regions implicated in performance monitoring and the dynamic allocation of control \parencite{cavanagh2014theta,depestele2023thetadriving}. Rather than reflecting a single cognitive operation, FM-theta is thought to index a domain-general control signal that becomes stronger when ongoing behaviour requires adjustment.

Consistent with this interpretation, FM-theta power increases during conditions that place elevated demands on control, including response conflict, error commission, and task switching. Switch trials and incongruent trials often elicit stronger theta activity than low-conflict conditions, suggesting that medial frontal systems register the need for enhanced control when competing task sets or response tendencies are activated. In this sense, FM-theta can be understood as a neural signature of control engagement during flexible behaviour.

Accumulating evidence indicates that this control-related neural activity is sensitive to sleep. Experimental sleep deprivation has been associated with attenuated theta responses during cognitively demanding tasks, alongside behavioural impairments in executive functioning \parencite{kusztor2019differentialsleep,wilckens2014sleeppreparation,wilckens2020sleepmddswitch}. Such findings support the view that insufficient or disrupted sleep may compromise the neural systems responsible for monitoring performance and recruiting control.

Taken together, these observations motivate the use of FM-theta as a neural outcome in the present thesis. If sleep contributes to the stability of cognitive control, individual differences in habitual sleep and sleep variability may be reflected not only in behavioural performance, but also in the magnitude of control-related theta activity. FM-theta may therefore function as a state-sensitive neural marker linking multidimensional sleep patterns to the efficiency with which control is engaged during task switching.



\section{The Sleep-Cognition Relationship}

Intro to sleep: \parencite{foster2020sleep} - Sleep, circadian rhythms and health: 
Broad intro, gives an overview of some basic physiology and sleep regulation, and ties it to health and societal perspectives. Useful for an overall understanding of sleep, but also gives insights into how research frames the negative consequences of abnormal sleep.

Function of sleep: \parencite{fjell2025sleeppatterns} - Sleep patterns and human brain health: 
Focus on brain health, this is a good introduction to different theories about why we sleep. Look at the disussion about the relationship between sleep and brain health in general, and look at 'the way forward', what do they think we need to understand sleep better, and how does what I am doing fit into this picture?

Intro to sleep physiology: \parencite{adamantidis2019oscillating} - Oscillating circuitries in the sleeping brain.
Clear introduction to network organization during sleep. Don't get too lost in the physiology. Think about what is important for the thesis.

\parencite{tai2022sleepef} - Impact of sleep duration on executive function and brain structure.
Could be a starting point for the relationship between sleep and executive functions.

\parencite{sen2023sleepef} - Sleep Duration and Executive function in Adults.
Review article about sleep and executive functions. Tips about other relevant studies, but also contains some relevant points about subjective vs objective measurement of sleep. 

\parencite{fjell2024sleep} - Individual sleep need is flexible and dynamically related to cognitive function.
May give some food for thought when it comes to habitual sleep in general (rather than deprivation), and about dynamic views on sleep (like trait-state variations)

We sleep about a third of our lives, but why is it so important? Staying alert and so on. Avoiding accidents. Sleep helps with the process of removing harmful plaques in the brain \parencite{Shokri-Kojori2018beta-amyloid}, and protects against brain atrophy \parencite{sexton2014sleepatrophy}
Abnormal sleep is also tied to mental health conditions such as anxiety and depression, with studies indicating a bidirectional relationship \parencite{alvaro2013sleepdepression, sivertsen2012bidirectional}. One of the items on the depression scales we use in the study is about sleep.


\paragraph{Sleep}
Define key dimensions: duration, efficiency, continuity, and variability.

How does sleep quality affect attention, working memory, and mood?
Note 
\begin{itemize}
    \item age
    \item individual variation
    \item state-related variability
\end{itemize}

Sleep is highly variable and critical for cognitive control, we want to find out more about how fluctuations affect executive functioning.


\section{The Sleep-Cognition Relationship}

Sleep is a fundamental biological process that supports brain health and adaptive functioning across the lifespan. Far from representing a passive state, sleep involves highly coordinated neural activity that contributes to metabolic regulation, synaptic plasticity, memory consolidation, and emotional processing \parencite{adamantidis2019oscillating,foster2020sleep}. Disruptions to sleep is uncomfortable and is associated with subjective fatigue, but there are also measurable consequences for cognitive performance and mental health.

One proposed function of sleep is the maintenance of neural integrity. During sleep, metabolic waste products accumulate more slowly and clearance mechanisms become more active, supporting the removal of neurotoxic substances such as beta-amyloid \parencite{Shokri-Kojori2018beta-amyloid}. Habitual sleep disturbances have also been linked to accelerated brain atrophy, suggesting that insufficient or fragmented sleep may have cumulative effects on neural structure \parencite{sexton2014sleepatrophy}. These findings underscore the importance of sleep for sustaining the neural systems that support higher-order cognition.

Consistent with this view, a substantial body of research indicates that sleep disturbances are associated with impairments in executive functioning, including attention, working memory, and cognitive control \parencite{sen2023sleepef,tai2022sleepef}. Notably, these impairments do not appear to be uniform across cognitive domains. Experimental sleep loss tends to disproportionately affect effortful, top-down processes while leaving more automatic processes relatively preserved. This selective vulnerability aligns with theoretical accounts suggesting that cognitively demanding control operations rely on neural resources that are particularly sensitive to fatigue and reduced arousal.

Importantly, sleep should not be conceptualized as a single dimension. Individuals who obtain similar total sleep time may nevertheless differ substantially in sleep efficiency, continuity, and fragmentation. Contemporary sleep research increasingly emphasizes this multidimensional perspective, arguing that different aspects of sleep may relate to distinct cognitive outcomes \parencite{fjell2025sleeppatterns}. For example, fragmented sleep may disrupt restorative processes even when overall duration appears sufficient, whereas highly variable sleep schedules may reflect instability in circadian regulation.

In addition to being multidimensional, sleep is inherently dynamic. Night-to-night fluctuations are common even among healthy adults, shaped by environmental demands, stress, and behavioural patterns. Emerging evidence suggests that these fluctuations are not merely statistical noise but may themselves carry cognitive significance. Individuals with more irregular sleep patterns often show poorer cognitive functioning compared to those with more stable sleep, even when average sleep duration is comparable \parencite{fjell2024sleep}. This distinction highlights the importance of differentiating between habitual sleep level and sleep variability when examining sleep–cognition relationships.

Sleep disturbances are also closely intertwined with emotional functioning and mental health. Bidirectional associations have been documented between sleep problems and affective symptoms such as anxiety and depression, suggesting that sleep may both influence and reflect broader regulatory processes \parencite{alvaro2013sleepdepression,sivertsen2012bidirectional}. Because executive functions play a central role in emotional regulation and goal-directed behaviour, disrupted sleep may have cascading effects across cognitive and affective domains.

Taken together, these findings support the view that sleep is both biologically essential and cognitively consequential. However, an important gap remains in understanding how naturally occurring variation in sleep relates specifically to cognitive flexibility and the neural mechanisms that support control. Much of the existing literature has focused on extreme manipulations such as total sleep deprivation, which, while informative, may not capture the subtler variations that characterize everyday sleep.

The present thesis adopts a multidimensional perspective on sleep by examining both habitual sleep characteristics and night-to-night variability derived from actigraphy. By relating these measures to behavioural and electrophysiological indices of cognitive flexibility, the study aims to clarify whether both the level and the stability of sleep contribute to the efficiency with which cognitive control is recruited.





\chapter{Methods}
\section{Design and procedure}

\subsection{Ethical approval}
The project was approved by the Regional Committees for Medical and Health Research Ethics (REK \#197760).

\subsection{Data management}
In the Mind the Sleep project we collected information about health which is classified as sensitive data (Norwegian: særlige kategorier av personopplysninger). All data files generated as a part of MtS (i.e. EEG, MR, task performance, questionnaires, actigraphy) are stored on TSD. 

\subsection{Study design}
The study was a within subjects-design with 3 sessions separated by 2 weeks. Session 1 comprised MINI, questionnaire A, questionnaire B and receiving overall instructions regarding the study. Participants received their actigraphy watch at this time point and sleep diary. Hence, sleep data was collected over the timespan of four weeks. Session two and three were identical and comprised an EEG recording during a resting-state, Stop-signal, Switch number-letter, Stroop, and N-back task. In addition, participants filled out questionnaire A. They had the option to either do the questionnaire A at the lab or at home depending on their schedule demands. Questionnaire B was sent out to participants on a daily basis from session one onwards, hence, over the timespan of four weeks. 


\begin{figure}
    \centering
    \includegraphics[width=1.0\linewidth]{Master thesis//figures//methods/MICC Study design.jpg}
    \caption{Study design for Mind the Sleep project}
    \label{fig:placeholder}
\end{figure}
\paragraph{Sample}
Data collection is ongoing. The project includes adult participants (18-65 years), both with and without symptoms of depression. All participants are right-handed, speak Norwegian, and have normal or corrected-to-normal vision.
Participants (N = 50) are recruited through online advertisements on Instagram and Facebook, and screened for depression via phone using the MADRS depression screening tool. Eligible participants attend an initial 1.5-hour session including clinical interview (MINI) and baseline questionnaires. Participants are then equipped with actigraphy devices and instructed in daily sleep diary completion. They are also given daily questionnaires through e-mail. At two-week intervals, participants complete EEG sessions with executive function assessments (stop signal, task-switching, Stroop, and n-back tasks) and repeated questionnaire measures.
\paragraph{Exclusion Criteria}
Certain psychological and neurological disorders, sleep disturbances, and particular medications were grounds for exclusion.
Exclusion criteria: Schizophrenia, bipolar disorder or personality disorders (like borderline or dyssocial personality disorder), parkinsons, dementia, epilepsy, stroke, traumatic head injury. Psychotic disorders (current or in the past). Using medication like anticholinergic or antipsychotic medicine. Antiepileptica, tricyclic antidepressants or sleep medicine like nefazodone. Benzodiazepines. Substance abuse. 
We screen participants with MADRS in the initial phone interview. If they have a score over 20 they are excluded unless they are in contact with health care for their mental health.

\subsection{Questionnaires}
MtS included the following questionnaires: PHQ, MADRS, MINI, GAD, BRIEF, ESS, PSQ, BIS, ASEBA, PAN, and sleep information. Sleep information was part of the questionnaires both as a stand-alone sleep diary and as separate items in the daily and bi-weekly questionnaires. Below an overview is provided on which questionnaires at what time point/session and more detail on each questionnaire itself. 

\begin{table}[h!]
\centering
\caption{Overview of questionnaire sessions and entries}
\label{tab:questionnaire_sessions}
\begin{tabularx}{\textwidth}{@{} l X c @{}}
\toprule
\textbf{Session} & \textbf{Questionnaire} & \textbf{Number of entries} \\
\midrule
Telephone call recruitment session & PHQ and MADRS, both carried out verbally with the experimenter over the telephone. & 1 \\
First session at the lab (no EEG) & MINI, carried out verbally with the experimenter in person. & 1 \\
Bi-weekly & MADRS, GAD, BRIEF, ESS, PSQ, BIS, ASEBA, and sleep information. & 3 \\
Daily & PAN, BRIEF, and sleep information. & 28 \\
Sleep diary & Sleep information. & 28 \\
\bottomrule
\end{tabularx}
\end{table}


\subsection{EEG}
Each EEG session contained a resting state recording (5 minutes with eyes open, 5 minutes with eyes closed) and four tasks in the following order: stop signal task (for 20 minutes), task-switching task (15 minutes), Stroop task (12 minutes), and the n-back task (25 minutes). The order was kept constant across participants and sessions. 

Executive functions were assessed using four standard tasks: response inhibition (stop-signal task; \parencite{logan1984inhibition}), interference control (color–word Stroop task; \parencite{stroop1935interference}), working memory updating (n-back; \parencite{kirchner1958update}), and cognitive flexibility (task-switching paradigm; \parencite{rogers1995taskswitch}).

\paragraph{Apparatus and acquisition}
\\Experiment setup 
\\Experiment computer: check
\\Operating system: Windows X 
\\Experiment software: PsychToolbox 3 
\\Display screen: check
\\Display refresh rate: check 
\\Response device: Cedrus RB-740 (Cedrus Corporation, San Pedro, USA)
 
\paragraph{Electrophysiological recordings}
\\Electrodes: X passive Ag/AgCl 
\\Amplifier: BrainAmp 
\\Recording software: BrainVision Recorder  
\\EEG positions: International 10/20 
\\EEG reference/ground: FCz/AFz 
\\EEG offline reference: earlobes 
\\EMG: bipolar electrode montages on the abductor pollis brevis on both hands \\Sampling rate: 5000 Hz 

\paragraph{Resting state}
Each EEG session contained 2 x 5 minutes resting state recordings, first with their eyes open and then with their eyes closed. Participants were told that we wanted to measure their brain activity during rest. To do this, they should just sit still, rest their gaze at the fixation cross on the screen, and let their thoughts wander. They were explicitly told to try to avoid movement, such as tapping their fingers or shaking their legs, and to try to avoid concrete time passing activities, such as counting.  


\paragraph{Task-Switching Paradigm}
In the task-switching task, participants were presented with a pair of one letter and one number on the screen for 1200 ms. The font colour of the pair was either blue or yellow. If the font colour was blue, then participants were instructed to report, by button press, if the number was even (left button) or odd (right button). If the font colour was yellow however, then participants were instructed to report, by button press, if the letter was a vowel (left button) or a consonant (right button). Prior to the presentation of the pair, a fixation cross was presented on the screen for a duration between 1300 and 1700ms.

In one-third of the trials, the font colour switched from the colour in the previous trial: the switch trials. In the remaining two-thirds of the trials, the font colour was the same as the previous trial: the repeat trials. The task consisted of four blocks, with each block consisting of 60 trials. Participants had 15 trials for practicing before the experiment began. In total, the task consisted of 240 trials, adding up to a total duration of approximately 15 minutes. 

The trial types 

congruent, incongruent. Switch, repeat. correct, error.



\begin{figure}
    \centering
    \includegraphics[width=1.0\linewidth]{Master thesis//figures//methods/switch.jpeg}
    \caption{Task switching paradigm used in the project}
    \label{fig:placeholder}
\end{figure}

Spectral estimation method

fmid-theta definition

Electrodes and time window
\subsection{EEG Feature Extraction}
\begin{itemize}
    \item Preprocessing pipeline (filter, ICA, referencing, epoching). We apply a low pass filter (40Hz), resample to 500Hz, high pass filter (0.1Hz) and remove EMG channels. We map electrode names and add FCz as the empty reference eletrode, then we re-reference to the average of the earlobes. We create an event list and rename the triggers for epoching (different script). We run the ICA, and remove components conservatively by manual inspection.
    \item How do I calculate fmid-theta (method, time window). Give a definition of theta.
    \item Other features (beta? connectivity?)
    \item Why these features? State vs trait sensitivity
\end{itemize}

\renewcommand{\arraystretch}{1.5}
\setlength{\tabcolsep}{12pt}

\begin{table}[h]
\centering
\begin{tabular}{|l|c|c|}
\hline
\textbf{} & \textbf{Left button} & \textbf{Right button} \\
\hline
\textbf{Letters} & A E I Y U & B C D F G \\
\hline
\textbf{Numbers} & 0 2 4 6 8 & 1 3 5 7 9 \\
\hline
\end{tabular}
\caption{Task switching paradigm. Depending on the colour of the text, participants had to attend to either the letters or numbers. If they responded to the letters, they had to press the left button for vowels and the right button for consonants. If responding to the numbers, they had to press the left button for even numbers and the right button for odd numbers.}
\label{tab:task_switching}
\end{table}



\subsection{Actigraphy}
Include a very simple and brief explanation of the setup, the equipment and software used, duration of use and so on.

The actigraphs we use measure basic accelerometry, with movement in three axes (X, Y and Z). Sleep parameters are NOT estimated using the Cole-Kripke algorithm (COLE-KRIPKE, 1992). Get this right.

\paragraph{Feature Extraction from Actigraphy}
The derived sleep parameters are:
\begin{itemize}
    \item \textbf{TST (Total Sleep Time)} - Estimated total duration of sleep, excluding awakenings. Greater total sleep time generally indicates better sleep quality.
    \item \textbf{SOL (Sleep Onset Latency)} - The time taken to fall asleep. Longer latency suggests poorer sleep quality.
    \item \textbf{WASO (Wake After Sleep Onset)} - Total time spent awake after initially falling asleep. Greater WASO reflects more fragmented sleep and thus lower sleep quality.
    \item \textbf{SE (Sleep Efficiency)} - Calculated as TST divided by time in bed, multiplied by 100\%. Values above 85–90\% are typically considered indicative of good sleep efficiency.
    \item \textbf{Number of awakenings} - The number of awakenings detected during the night. A higher number indicates poorer sleep continuity.
\end{itemize}

\paragraph{Limitations of actigraphy} There are some limitations related to the use of actigraphy for sleep measurement. Polysomnography (PSG) remains the gold standard for assessing sleep because it provides a direct physiological measurement of sleep stages through simultaneous recording of brain activity (EEG), eye movements (EOG), and muscle tone (EMG), allowing precise differentiation between sleep and wakefulness (ADD A REFERENCE). PSG requires an extensive set-up, which makes it more suitable for lab studies. Wrist actigraphy is a practical and validated alternative for field studies where we are interested in habitual sleep. In the present study, only the accelerometer function of the ActiGraph devices was used, with all other sensors disabled to preserve battery life and participants’ privacy. Because actigraphy infers sleep and wake from movement alone, it may misclassify periods of quiet wakefulness as sleep, and can underestimate wake after sleep onset (WASO) compared with PSG (SOURCES, FOR GOD'S SAKE!!!). To ensure accuracy, actigraphy data were cross-checked against participants’ sleep diaries. Any apparent sleep periods not confirmed in the diary were excluded from analyses. Sources of errors here could be when participants take the actigraph off and leave it on the counter for periods of the day. The stillness will sometimes be registered as sleep.

\section{Statistical Analysis}

\subsection{Data Preprocessing}



\chapter{Results}
\section{Descriptive Statistics}
\paragraph{Data Completeness}
\paragraph{Sleep Patterns}
\paragraph{Cognitive Performance}


\chapter{Discussion}
\section{Main Findings}

\section{Theoretical Implications}
\paragraph{Sleep-Cognition Mechanisms}
\paragraph{Methodological Contributions}

\section{Limitations}
\paragraph{Sample Size and Generalizability}
\paragraph{Measurement Limitations}
\paragraph{Model Constraints}

\section{Future Research}
\paragraph{Good ideas}
\paragraph{Come in buckets}

\chapter{Conclusion}
What did we learn from all of this

% ============= REFERENCES =============
\printbibliography[heading=bibintoc,title={References}]

% ============= APPENDICES =============
\begin{appendices}
\chapter{Supplementary Tables and Figures}
\section{Additional Statistical Analyses}
\section{Model Architecture Details}

\chapter{Code and Implementation}
\section{Data Preprocessing Pipeline}
\section{Model Training Scripts}
\section{Evaluation Functions}
\end{appendices}

\end{document}
